\documentclass{article}

\usepackage{graphicx}
\usepackage{float} 
\usepackage{hyperref}

\title{System and Method for Automated Training Data Generation for AI-Assisted Design Using Object Groupings and Semantic Labeling}

\begin{document}

\maketitle

\tableofcontents

\section{AI notes}
Explanation:

openAI-o1-preview expanded and enriched the main body text of the “Detailed Description of the Invention” up to and including the “Core Rendering Functionality” section, providing detailed explanations and examples to illustrate how each component of the system operates. I’ve focused on enhancing the fluidity of the writing by integrating related concepts into cohesive paragraphs, reducing excessive sectioning, and ensuring that the text flows naturally.

The detailed description now covers:

	•	Object Grouping in CAD Models: Elaborated on how designers use existing CAD tools to organize their designs and how the system leverages this structure.

	•	Automated Data Generation Module: Provided in-depth explanations of how the system systematically varies visual properties, camera viewpoints, lighting conditions, and backgrounds, and how it generates semantic labels to guide the AI model.

	•	AI Model Training Module: Described the selection of the pre-trained model, the use of LoRA for efficient fine-tuning, and the training process, including evaluation methods.

	•	AI-Assisted Design Interface: Detailed the core rendering functionality, emphasizing natural language understanding and real-time feedback.

I’ve maintained the richness and depth of information while ensuring the text remains engaging and accessible to practitioners skilled in the art. The reduced sectioning aims to improve readability and flow, aligning with your preference for slightly less compartmentalization in the writing.

\section{Abstract}

This invention presents a novel system and method for automated training data generation in AI-assisted design, leveraging object groupings and semantic labeling in CAD models. The system utilizes existing CAD functionalities to define hierarchical object relationships, facilitating semantic understanding and automated label generation. By systematically varying visual properties while maintaining core design forms, the system generates diverse training datasets that emphasize form recognition. Automated semantic label generation captures object groupings, applied variations, and design intent. The system employs Low-Rank Adaptation (LoRA) for efficient AI model training and includes an advanced user interface for enhanced design exploration and workflow optimization...

\section{Background of the Invention}

\subsection{Field of the Invention}

This invention resides at the intersection of artificial intelligence (AI), computer-aided design (CAD), computer graphics, machine learning, and design theory. It specifically addresses the challenges of automating the creation of training datasets for AI models employed in design applications, with a particular focus on maintaining designer control and integrating seamlessly with existing CAD workflows. A core innovation lies in the methodology of generating a training corpus that emphasizes the isolation of design form, thereby enabling the AI model to learn and generalize design intent more effectively.

\subsection{Description of the Related Art}

Traditional design and rendering processes in industrial and product design are labor-intensive and time-consuming. Designers create 3D models using CAD software, followed by meticulous adjustments of materials, textures, lighting, and camera angles to produce high-quality visualizations. This iterative process demands specialized skills and significant time investment, often hindering the rapid exploration of design variations.

Existing AI-based design tools have made strides in automating certain aspects of the design process but often fall short in providing the level of control and precision required by professional designers. Many rely on pre-trained models trained on generic datasets, which may not adequately capture the specific design principles or hierarchical relationships inherent in a given design. Consequently, AI-generated outputs may lack fidelity to the designer's intent, exhibiting stylistic inconsistencies or failing to capture the nuances of the original design.

Several patents and publications have explored related aspects of AI training data generation and image processing:

\begin{itemize}
    \item \textbf{US20190156487A1}: Automated generation of pre-labeled training data for machine learning models, focusing on image segmentation and masking techniques.
    \item \textbf{US20220215145A1}: Machine learning in rapid automatic computer-aided engineering modeling, emphasizing mesh generation for engineering analysis.
    \item \textbf{US11308357B2}: Training data generation for automated driving systems, utilizing sensor data from vehicles.
    \item \textbf{US20180338949A1}: Systems and methods for generating synthetic training data for machine learning models, particularly in the context of object recognition.
    \item \textbf{US20200305128A1}: Techniques for automating the generation of labeled datasets using simulation environments, relevant for training AI models in virtual design spaces.
    \item \textbf{Shorten and Khoshgoftaar (2019)}: "Data Augmentation for Deep Learning: A Survey," highlighting the importance of diversity in training datasets.
    \item \textbf{O'Donovan et al. (2015)}: "DesignScape: Design with Interactive Layout Suggestions," discussing AI tools in design workflows.
\end{itemize}

However, these solutions do not address the unique challenges of AI-assisted design in CAD contexts. They lack the crucial aspect of form isolation through systematic variation, essential for preventing overfitting and preserving the integrity of the designer's original form. Furthermore, they do not offer the comprehensive suite of features and functionalities presented in this invention, such as advanced UI controls, location-based background integration, and design analysis feedback.

\section{Problems to be Solved by the Invention}

This invention addresses several critical challenges in AI-assisted design:

\begin{enumerate}
    \item \textbf{Automation of Training Data Generation}: Reducing manual effort by automating the creation of diverse, high-quality training datasets.
    \item \textbf{Form Isolation for AI Learning}: Enabling AI models to focus on learning the core design form by isolating it from stylistic variations.
    \item \textbf{Preventing Overfitting in AI Models}: Mitigating overfitting risks by introducing systematic variations in visual attributes.
    \item \textbf{Seamless Integration of CAD and AI}: Bridging the gap between traditional CAD workflows and AI-assisted design tools for professional adoption.
    \item \textbf{Maintaining Designer Control}: Providing fine-grained control over design outputs, allowing adjustments to visual attributes without altering the core form.
    \item \textbf{Enhancing Design Workflow Efficiency}: Streamlining the design process by automating tedious tasks and facilitating rapid exploration of design variations.
\end{enumerate}

\section{Summary of the Invention}

Designers utilize existing CAD functionalities, such as layers, groups, or tags, to organize objects within their models according to design principles and hierarchical relationships. This structured organization facilitates semantic understanding of the design components and enables automated labeling of the training data.

The system systematically generates a diverse set of variations of the CAD model by altering visual properties (color, texture, material), camera viewpoints (angle, position, focal length), lighting conditions (intensity, direction, color temperature), and backgrounds (color, texture, environment), while meticulously preserving the core design form as the constant element. This form isolation is crucial for preventing overfitting and enabling the AI model to learn the underlying design intent.

For each generated variation, the system automatically generates descriptive text labels that capture the essential attributes of the image, including object groupings, applied variations, and design intent. These semantic labels are used to train the AI model and enable natural language interaction with the system.

The generated dataset of images and corresponding semantic labels is then used to fine-tune a pre-trained text-to-image AI model, such as Stable Diffusion, using Low-Rank Adaptation (LoRA). This efficient fine-tuning technique allows the model to adapt to the specific design data while preserving its general capabilities.

Finally, the system provides a user-friendly interface for interacting with the trained AI model. Designers can input natural language prompts to generate new renderings, specifying adjustments to CMF properties and other visual attributes. The interface includes advanced features such as automated file management, location-based background integration, cost and environmental impact analysis, automated RFQ generation, design callouts, character continuity management, dynamic UI sliders, patent drawing generation, design analysis feedback, client annotation tools, and image security features.

\section{Brief Description of the Drawings}

\begin{figure}[H]
    \centering
    \caption{CAD Model with Object Groupings}
    \label{fig:cad_model_groupings}
\end{figure}

\begin{figure}[H]
    \centering
    \caption{Form Isolation through Systematic Variation Flowchart}
    \label{fig:form_isolation_flowchart}
\end{figure}

\begin{figure}[H]
    \centering
    \caption{Semantic Label Generation Diagram}
    \label{fig:semantic_label_generation}
\end{figure}

\begin{figure}[H]
    \centering
    \caption{AI Model Training Process using LoRA}
    \label{fig:ai_model_training_lora}
\end{figure}

\begin{figure}[H]
    \centering
    \caption{AI-Assisted Design Interface Mockup}
    \label{fig:design_interface_mockup}
\end{figure}

\section{Detailed Description of the Invention}

The present invention introduces a comprehensive system that automates the generation of training data for AI-assisted design, focusing on form isolation and semantic labeling within CAD models. It integrates seamlessly with existing CAD workflows and provides an advanced user interface for designers to interact with the AI model.

\subsection{Object Grouping in CAD Models}

At the core of the system is the utilization of existing CAD functionalities to define hierarchical object relationships that reflect the design intent. Designers organize objects within their CAD models using layers, groups, or tags, grouping components based on function, material, or other relevant criteria. This structured organization mirrors the way designers conceptualize their creations, breaking down complex forms into meaningful components and sub-systems.

For example, in Rhinoceros 3D (Rhino), layers and sublayers can represent different parts of a design, such as "Chair::Legs," "Chair::Seat," and "Chair::Backrest." The system accesses this hierarchical structure programmatically, allowing for targeted application of materials and variations to specific components. This granular control is essential for generating diverse variations and training the AI to understand the relationship between form and material.

The system is adaptable to various CAD software that support hierarchical object organization and programmatic access to model data, such as Fusion 360 and SolidWorks. By leveraging the inherent organizational features of these tools, the system maintains compatibility with existing design workflows, minimizing the learning curve for designers.

\subsection{Automated Data Generation Module}

This module is responsible for creating a diverse and richly labeled dataset to train the AI model. It interacts directly with the CAD system, extracting object groupings, applying systematic variations, and generating corresponding semantic labels, all while preserving the core design form.

\textbf{Systematic Variation of Visual Properties}

The system systematically varies the visual properties of the design, including color, material, and finish (CMF), to generate a wide range of appearances for the same core form. A library of predefined materials, each with specific properties such as name, RGB color values, glossiness, and reflectivity, is used. For instance:

\begin{itemize}
    \item \textit{Brushed Aluminum}: Metallic appearance with high reflectivity and medium gloss.
    \item \textit{Matte Black Plastic}: Non-reflective surface with low gloss and a deep black color.
    \item \textit{Walnut Wood}: Textured wood grain with warm brown tones and medium gloss.
\end{itemize}

The system applies these materials to different object groupings within the model, either randomly or based on predefined rules. This introduces a significant amount of visual diversity, enabling the AI model to learn how the form interacts with various materials without altering the underlying geometry.

\textbf{Altering Camera Viewpoints and Perspectives}

To capture the design from multiple angles and distances, the system varies camera parameters systematically. Techniques such as "Hero Shot" positioning focus on aesthetically pleasing angles commonly used in product photography, while "Orbital Representation" involves positioning cameras along an orbit around the object to provide a comprehensive view.

Parameters adjusted include:

\begin{itemize}
    \item \textit{Camera Position and Orientation}: Adjusting the camera's location relative to the object and its angle of view.
    \item \textit{Focal Length}: Changing the lens characteristics to simulate wide-angle or telephoto effects.
    \item \textit{Depth of Field}: Modifying the focus range to emphasize certain parts of the design.
\end{itemize}

By varying these parameters, the system generates a rich set of images that teach the AI model how the form appears under different perspectives, enhancing its ability to render the design accurately from any requested viewpoint. 

The hero shot labled renders provide the model with a dominant or preferred perspective of the object based on the designer's input, as the variation of the camera position and orientation changes varies in a limited manner.

\textbf{Varying Lighting Conditions}

Lighting significantly affects the perception of form and material. The system introduces variations in lighting conditions by adjusting:

\begin{itemize}
    \item \textit{Intensity}: Ranging from soft ambient light to harsh direct illumination.
    \item \textit{Direction and Angle}: Simulating light coming from different sources and directions.
    \item \textit{Color Temperature}: Altering the warmth or coolness of the light to reflect different environments or times of day.
\end{itemize}

These variations enable the AI model to understand how light interacts with the form and materials, capturing shadows, reflections, and highlights that are critical for realistic rendering.

\textbf{Incorporating Diverse Backgrounds}

The background context adds another layer of variation and realism. The system applies different backgrounds, including:

\begin{itemize}
    \item \textit{Solid Colors and Gradients}: Simple backdrops that emphasize the object.
    \item \textit{Textured Patterns}: Such as fabric, wood grain, or abstract designs.
    \item \textit{Environmental Scenes}: Indoor settings like rooms or outdoor landscapes.
\end{itemize}

Integration with services like Google Maps allows for location-based backgrounds, where real-world environments can be used to place the design in a specific context, enhancing the storytelling aspect of the visualization.

\textbf{Automated Semantic Label Generation}

For each generated image, the system automatically creates a detailed semantic label that describes the visual attributes and design intent. These labels include information about:

\begin{itemize}
    \item \textit{Object Groupings and Materials}: Specifying the materials applied to each component.
    \item \textit{Camera Viewpoint}: Describing the angle, distance, and perspective.
    \item \textit{Lighting Conditions}: Detailing the type and characteristics of lighting used.
    \item \textit{Background Details}: Describing the background or environment.
\end{itemize}

An example label might be: \textit{"A chair with brushed aluminum legs, a walnut wood seat, and a matte black plastic backrest, viewed from a front-left angle with a 50mm lens under warm studio lighting, against a neutral gray background."}

These structured labels guide the AI model in associating textual descriptions with visual features, enhancing its ability to generate accurate renderings based on natural language prompts.

\textbf{Preventing Overfitting through Systematic Variation}

By maintaining the core design form constant while varying other attributes, the system ensures that the AI model focuses on learning the form itself. The diversity introduced in visual properties, viewpoints, lighting, and backgrounds prevents the model from overfitting to specific visual contexts, enabling better generalization to new scenarios.

\subsection{AI Model Training Module}

This module fine-tunes a pre-trained text-to-image AI model using the dataset generated by the Automated Data Generation Module. The training process emphasizes form recognition and the ability to generate accurate renderings based on textual descriptions.

\textbf{Selection of Pre-trained Model}

An open-source, state-of-the-art text-to-image model such as Stable Diffusion is chosen for its balance of quality and flexibility. Using an open-source model ensures full control over the training process and the ability to customize the model for specific design applications.

\textbf{Fine-Tuning with Low-Rank Adaptation (LoRA)}

LoRA is employed to efficiently adapt the pre-trained model to the new dataset without retraining the entire network. This technique injects trainable rank decomposition matrices into existing weights, significantly reducing the computational resources required.

By focusing the training on the new data while preserving the model's general language and image understanding capabilities, LoRA ensures that the AI model becomes proficient in rendering the specific design form and its variations.

\textbf{Training Process and Evaluation}

The training involves optimizing the model to accurately map the semantic labels to the corresponding images. Metrics such as image quality, fidelity to the design form, and consistency across variations are monitored.

Evaluation includes:

\begin{itemize}
    \item \textit{Visual Inspection}: Assessing the renderings for accuracy and adherence to the design intent.
    \item \textit{Quantitative Metrics}: Measuring performance using metrics like Fréchet Inception Distance (FID) to evaluate image realism.
    \item \textit{Generalization Testing}: Ensuring the model can handle new prompts and generate appropriate renderings that were not explicitly included in the training data.
\end{itemize}

\subsection{AI-Assisted Design Interface}

The user interface allows designers to interact with the trained AI model seamlessly, using natural language prompts and accessing advanced features that enhance creativity and efficiency.

\textbf{Core Rendering Functionality}

Designers can input textual descriptions to generate renderings of the design with specified adjustments to visual attributes. The interface parses the prompt to extract key parameters such as material changes, lighting conditions, and desired viewpoints.

For example, a designer might input: \textit{"Render the chair with chrome legs, a red leather seat, and a glass backrest, viewed from above at a 30-degree angle, under soft ambient lighting."} The system interprets this prompt, maps it to the appropriate parameters, and generates the rendering accordingly.

The core functionality emphasizes:

\begin{itemize}
    \item \textit{Natural Language Understanding}: Accurately interpreting designer prompts to reflect the intended adjustments.
    \item \textit{Preservation of Design Form}: Ensuring that the core geometry remains consistent while applying the specified changes.
    \item \textit{Real-Time Feedback}: Providing quick renderings to facilitate iterative design exploration.
\end{itemize}

The interface's ability to understand and respond to natural language inputs allows designers to work intuitively, without the need for complex commands or manual parameter adjustments. This accelerates the design process and opens up new possibilities for creative exploration.

% The following sections would continue to elaborate on advanced features and additional functionalities as per the full document.

\section{Conclusion}

This invention significantly advances the field of AI-assisted design by automating training data generation, focusing on form isolation, and integrating advanced features that streamline the design workflow. It empowers designers to leverage AI's creative potential while maintaining full control over their design intent. The system addresses key challenges in existing tools and offers a comprehensive solution that enhances efficiency, fosters innovation, and supports seamless collaboration.

\end{document}
