\documentclass{article}

\usepackage{graphicx}
\usepackage{hyperref}

\title{System and Method for Automated Training Data Generation for AI-Assisted Design Using Object Groupings and Semantic Labeling}

\begin{document}


\begin{abstract}
   This invention describes a system and method for automating the generation of training data tailored for AI-assisted design applications. The system leverages the hierarchical structure of objects within CAD models, generating diverse variations through systematic manipulation of visual properties, camera viewpoints, lighting, and backgrounds, while critically maintaining the core design form. This "form isolation" technique addresses a key challenge in training AI models for design: preventing overfitting to irrelevant aesthetic qualities. By decoupling form from style, the AI learns to generalize design intent more effectively. The system automatically generates descriptive semantic labels for each variation, facilitating the training of a text-to-image AI model. A sophisticated user interface empowers designers to generate renderings via natural language prompts, offering fine-grained control over CMF properties and other visual attributes while preserving design integrity. Advanced features such as location-based backgrounds, automated cost analysis, design callouts, and character continuity further streamline the design workflow and enhance collaborative capabilities.
\end{abstract}

\section{Background of the Invention}

\subsection{Field of the Invention}

This invention lies at the confluence of artificial intelligence (AI), computer-aided design (CAD), computer graphics, machine learning, and design theory, specifically addressing the automated creation of training datasets for AI-driven design applications. The core innovation lies in the methodology of generating a training corpus that emphasizes the isolation of design form, thereby enabling the AI model to learn and generalize design intent more effectively.

\subsection{Description of the Related Art}

Traditional design and rendering workflows are resource-intensive, requiring significant manual effort and specialized skills. Existing AI-assisted design tools, while promising, often lack the precision and control required by professional designers, particularly regarding the preservation of design intent and the ability to manipulate specific attributes like CMF properties independently of form.

Previous attempts to automate training data generation for AI models have focused on general image processing techniques (e.g., US20190156487A1, image segmentation and masking) or domain-specific applications like computer-aided engineering (US20220215145A1, mesh generation) and automated driving systems (US11308357B2, sensor data). However, these approaches do not address the unique challenges inherent in AI-assisted design, specifically the need to isolate and preserve the core design form during the training process. They also lack the comprehensive feature set and integrated workflow offered by this invention. A key differentiator of this invention is the explicit focus on form isolation, achieved by systematically varying all other visual elements except the core form. This approach directly addresses the problem of overfitting, a common challenge in machine learning where a model becomes too specialized to the training data and fails to generalize to new examples. By training the AI on a dataset where the form remains constant while all other attributes vary, the model learns to recognize and generate the form independently of specific styles, colors, textures, or viewpoints. This is crucial for enabling designers to manipulate these attributes freely through natural language prompts without altering the fundamental design. The provided code snippet exemplifies this principle by demonstrating the random application of materials and lighting to different parts of a CAD model (defined by sublayers), while the underlying geometry of the model remains unchanged. This code forms a part of the Automated Data Generation Module, which is a core component of this invention. It demonstrates how the system programmatically generates variations in visual properties to create a diverse training dataset.

\section{Problems to be Solved by the Invention}

This invention directly addresses several crucial limitations of current AI-assisted design technologies:

\begin{itemize}
    \item Overfitting to Irrelevant Aesthetic Qualities: Existing AI models trained on conventional datasets often overfit to specific styles, colors, textures, lighting, or camera angles, hindering their ability to generalize design intent and limiting the designer's control over these attributes.
    \item Manual and Time-Consuming Rendering Workflows: Traditional rendering processes are labor-intensive and slow, hindering rapid design exploration and iteration.
    \item Lack of Seamless CAD and AI Integration: Current AI design tools often exist as separate entities, requiring cumbersome data transfer and limiting the integration with established CAD workflows.
    \item Insufficient Designer Control over Generated Outputs: Many AI tools lack the fine-grained control that designers need to manipulate specific design attributes while preserving the overall form and intent.
    \item Limited Collaboration and Feedback Capabilities: Traditional design review processes can be inefficient, lacking tools for streamlined feedback and collaborative iteration.
    \item Lack of Design Analysis and Optimization: Existing AI tools primarily focus on generating visually appealing outputs but lack the ability to analyze designs for potential functional or manufacturing issues.
\end{itemize}

\section{Summary of the Invention}

This invention provides a system and method for automating the generation of training data for AI-assisted design applications, focusing on form isolation to prevent overfitting and enhance designer control. The key components and functionalities include:

\begin{itemize}
    \item Object Grouping in CAD Models: Leveraging existing CAD functionalities (layers, groups, tags) to define hierarchical object relationships that reflect design intent. This structured organization facilitates semantic understanding and automated label generation. This structure mirrors the way designers think about their creations, breaking down complex forms into meaningful components and sub-systems. The code example provided demonstrates how object groupings (represented by sublayers in Rhino) are used to apply materials to specific parts of the model. This granular control over material application is essential for generating diverse variations and training the AI to understand the relationship between form and material.

    \item Form Isolation through Systematic Variation: Generating diverse variations of the CAD model by systematically altering visual properties (CMF), camera viewpoints, lighting conditions, and backgrounds while maintaining the core design form as the invariant element. This form isolation is the central innovation, enabling the AI model to learn the form independently of other visual attributes and preventing overfitting. The code snippet illustrates this by randomly assigning materials and lighting to the different parts of the chair model, effectively creating numerous variations while the chair's shape remains constant. This process, applied across a wide range of parameters, generates a robust training dataset that emphasizes form recognition.

    \item Automated Semantic Label Generation: Automatically creating descriptive text labels for each generated variation, capturing object groupings, applied variations, and design intent. These labels provide the textual context necessary for training the AI model and facilitate natural language interaction with the system. The code example shows how these labels are constructed by combining the class name (e.g., "chair"), instance name, viewpoint description, lighting description, and material descriptions. This structured approach ensures consistency and clarity in the labels, allowing the AI to effectively associate text prompts with specific visual attributes.

    \item AI Model Training using LoRA: Fine-tuning a pre-trained text-to-image AI model (e.g., Stable Diffusion) using Low-Rank Adaptation (LoRA). This efficient method adapts the model to the specific design data without retraining the entire model, preserving computational resources and accelerating the training process. LoRA is particularly well-suited for this application as it allows the model to learn the nuances of the specific design while retaining the general image generation capabilities of the pre-trained model.

    \item AI-Assisted Design Interface with Advanced Features: A user-friendly interface enabling designers to interact with the trained AI model through natural language prompts. This interface includes advanced features that streamline the design workflow and enhance creative exploration:

    \item Automated File Management (Naming, Storage, Archiving): Automates file organization and storage in the cloud, simplifying asset management and facilitating collaboration.

    \item Detailed Close-Ups: Generates high-resolution close-up renders of specific design elements.

    \item Location-Based Backgrounds (Google Maps Integration, Depth Maps, ControlNet): Incorporates real-world locations as backgrounds, enhancing realism and context.

    \item Mass \& Volume Calculations (Cost Estimation, Environmental Impact, Automated RFQs): Provides valuable data for design optimization and cost analysis.

    \item Design Callouts (Automated Generation, Editable Annotations): Automates the creation of professional-style design callouts.

    \item Character Continuity (Face Swapping, Consistent Descriptions): Ensures consistent character appearance across multiple renders.

    \item Dynamic UI Sliders (Context-Sensitive Parameters): Provides intuitive control over rendering parameters based on the prompt.

    \item Patent Drawing Generation: Automates the creation of patent drawings from the CAD model.

    \item Design Analysis (GPT Integration, Potential Problem Highlighting): Offers valuable feedback on design feasibility and potential issues.

    \item Client Annotation Tools (Collaborative Review and Feedback): Facilitates collaborative review and feedback directly on rendered images.

    \item Image Security (Watermarking, Serialization): Protects intellectual property and tracks image usage.

    \item Vision Tracking (Gaze Analysis, Heatmaps): Provides insights into viewer attention patterns.

    \item Modular UI (Drag-and-Drop, Texture Blending): Offers a flexible and customizable interface for applying materials and styles.

    \item  Style References (Style Transfer, Dynamic Style Parameters): Allows designers to apply different artistic styles to their renders.

    \item Thematic Sliders (Mood and Style Control): Provides control over the overall mood and aesthetic of the render.

    \item Preset Management (Saving, Loading, Sharing): Enables efficient reuse of rendering settings and styles.

    \item Export Options (3D Printing, Foldable Models, etc.): Supports various output formats for different applications.

    \item History Tree and Visual Diff: Tracks design iterations and visually compares variations.

\end{itemize}

\section{Brief Description of the Drawings}

Figures 1-10 as described previously, potentially adding more diagrams to illustrate specific UI elements and functionalities


\section{Outline of the Description of the Invention}

\paragraph{Directions for the Detailed Description outline}

This section provides a comprehensive explanation of the invention, elaborating on the functionalities of each component and the overall workflow.  It will reference the provided code examples to illustrate the implementation of specific features and explain how the system addresses the challenges outlined in the previous sections.

\subsection{System Components}
\paragraph{A. CAD System}  
\begin{itemize}
    \item Supported CAD Software:  List specific compatible software (e.g., Rhinoceros 3D, Fusion 360, SolidWorks) and the rationale for their selection (e.g., feature set, scripting capabilities).
    \item Object Grouping Mechanisms: Explain how object groupings are represented in different CAD systems (e.g., layers in Rhino, components in Fusion 360) and how these mechanisms are leveraged by the system.
    \item Importing and Preprocessing CAD Data: Detail the process of importing CAD models into the system and any necessary preprocessing steps (e.g., mesh conversion, geometry simplification).
\end{itemize}
\paragraph{B. Automated Data Generation Module}

B. Automated Data Generation Module
\subparagraph{Modularity}
\begin{itemize}
    \item Identifying Object Groupings: Describe the algorithms and methods used to automatically identify object groupings within the CAD model based on user-defined criteria or pre-defined rules. Reference the code example where sublayers are used to define object groups.
    \item Generating Variations:
\end{itemize}
\subparagraph{Variation}
\begin{itemize}
    \item Visual Properties (CMF): Detail the process of systematically varying color, material, and finish properties. Explain how randomness and predefined ranges are used, referencing the \texttt{MATERIAL\_LIST} and \texttt{create\_random\_material()} function from the code.
    \item Camera Viewpoints: Explain how camera positions, angles, orientations, and focal lengths are varied. Reference the \texttt{generate\_camera\_positions()} and \texttt{generate\_viewpoint\_description()} functions from the code, explaining how hero shot and orbital representation modes are implemented.
    \item Lighting Conditions: Describe the process of generating diverse lighting scenarios, including variations in intensity, direction, color temperature, and type of light source. Reference the \texttt{generate\_lighting\_variation()} and \texttt{apply\_lighting\_to\_render()} functions.
    \item Backgrounds: Explain how different backgrounds are applied to the rendered images, including solid colors, gradients, textures, patterns, and environmental contexts. Discuss the integration with Google Maps for location-based backgrounds, using depth maps and ControlNet.
    \item Semantic Label Generation: Detail the process of automatically generating descriptive text labels for each variation. Explain how the labels incorporate information about object groupings, applied variations, and design intent. Reference the \texttt{generate\_caption()} function and explain how it combines different elements to create a comprehensive caption.
    \item Data Augmentation Techniques: Discuss any additional data augmentation techniques used to further diversify the training dataset (e.g., random cropping, rotation, noise addition).
\end{itemize}


\paragraph{C. AI Model Training Module}
      1. Pre-trained Model Selection: Justify the choice of the pre-trained text-to-image AI model (e.g., Stable Diffusion) and explain its advantages in the context of design.
      2. Fine-tuning with LoRA: Detail the process of fine-tuning the pre-trained model using Low-Rank Adaptation (LoRA).  Explain the benefits of LoRA in terms of efficiency and targeted adaptation.
      3. Training Process and Hyperparameter Tuning: Describe the training process, including data preparation, batch size, learning rate, and number of epochs.  Explain how hyperparameters are tuned to optimize model performance.
      4. Evaluation Metrics:  Discuss the metrics used to evaluate the performance of the trained AI model (e.g., image quality, fidelity to design intent, diversity of generated outputs).

\paragraph{D. AI-Assisted Design Interface}
   **D. AI-Assisted Design Interface**

      1.  **Core Rendering Functionality:** Explain how designers use natural language prompts to generate renderings. Discuss the parsing of prompts and the translation of text descriptions into rendering parameters.

      2.  **Advanced Features:**  Provide a detailed explanation of each advanced feature, including implementation details and user interaction:
         a. Automated File Management (Naming, Cloud Storage, Archiving): Discuss the implementation details and benefits of automated file organization, cloud storage integration, and version control.
         b. Detailed Close-Ups: Explain how users can specify regions of interest for generating high-resolution close-ups.
         c. Location-Based Backgrounds: Detail the integration with Google Maps API, depth map generation, and the use of ControlNet for precise background placement and control.
         d. Mass \& Volume Calculations:  Explain how the system calculates mass, volume, estimated cost, and environmental impact based on CMF properties and CAD data.  Describe the automated RFQ generation process.
         e. Design Callouts: Detail the process of automatically generating design callouts using image segmentation and prompt generation.  Explain how users can edit and customize the callouts.
         f. Character Continuity:  Explain the techniques used to maintain character consistency across renders, including face swapping and text-based character description analysis.
         g. Dynamic UI Sliders: Describe how sliders are dynamically generated based on the prompt context and how they provide intuitive control over rendering parameters.
         h. Patent Drawing Generation: Detail the process of automatically generating patent drawings from the CAD data, including line art generation and annotation placement.
         i. Design Analysis:  Explain the integration with a GPT-based model for design analysis.  Describe how potential problems and areas for improvement are identified and presented to the user.
         j. Client Annotation Tools: Describe the collaborative annotation features, allowing clients and reviewers to provide feedback directly on rendered images.
         k. Image Security:  Explain the watermarking and serialization techniques used to protect intellectual property and track image usage.
         l. Vision Tracking: Detail the implementation of eye-tracking analysis and the generation of heatmaps to visualize user attention patterns.
         m. Modular UI:  Describe the drag-and-drop interface, texture blending capabilities, and customization options.
         n. Style References: Explain how users can apply different artistic styles to their renders and how style parameters are controlled.
         o. Thematic Sliders: Discuss the implementation of sliders for controlling mood and overall style.
         p. Preset Management: Detail the preset management system, allowing users to save, load, and share rendering configurations.
         q. Export Options:  List the supported export formats and their respective applications.
         r. History Tree and Visual Diff: Explain how the system tracks design iterations and provides visual comparison capabilities.


\paragraph{II. Process Steps}    

Provide a detailed walkthrough of the entire process, from importing the CAD model to generating final renderings, referencing the specific functionalities of each component described in Section I.

\begin{itemize}
   \item Preparing the CAD Model: Importing the CAD model, defining object groupings, and setting up any necessary preprocessing steps.
   \item Generating the Training Dataset: Running the Automated Data Generation Module to create variations and semantic labels.  Illustrate this with specific examples and reference the code provided.
   \item Training the AI Model:  Fine-tuning the pre-trained model using the generated dataset and LoRA.  Discuss the training parameters and evaluation metrics.
   \item Generating Renderings with the AI-Assisted Design Interface:  Demonstrate the use of natural language prompts and the various advanced features of the interface.  Provide specific examples and illustrate the workflow with screenshots or mockups.
\end{itemize}



\section{Detailed Description of the Invention}

\subsection{I. System Components}

\paragraph{A. CAD System}

The system is designed to be compatible with a range of industry-standard CAD software, enhancing its accessibility and integration into existing design workflows. While the principles of the invention apply broadly, specific implementations may leverage particular features of different CAD packages. The following describes the supported software and their relevance to the invention:

Supported CAD Software: The system is designed for compatibility with CAD software packages that allow for hierarchical object organization and programmatic access to model data. These features are crucial for automating the variation generation and semantic labeling processes. Specifically, the current implementation supports:

Rhinoceros 3D (Rhino): Rhino is chosen for its robust scripting capabilities using Python, enabling deep integration with the Automated Data Generation Module. The provided code examples demonstrate this integration, utilizing Rhino's Python API (rhinoscriptsyntax) to access and manipulate model data, such as object layers, materials, and rendering settings. Rhino’s widespread use in industrial design and its open architecture make it an ideal platform for developing and deploying this invention.

Autodesk Fusion 360: Fusion 360 is another suitable CAD platform due to its parametric modeling capabilities and API access. Its cloud-based nature facilitates collaborative workflows and integration with other Autodesk services. While not explicitly demonstrated in the provided code, the principles of the invention can be readily applied to Fusion 360 using its Python API.

SolidWorks: SolidWorks, with its extensive use in engineering and product design, is also a target platform for this invention. Its API, while primarily supporting C++, can be accessed through COM interop from Python, enabling similar functionalities as in Rhino and Fusion 360.

The system is not limited to these specific CAD packages. Any CAD software offering similar functionalities in terms of object hierarchy and programmatic access can be integrated with the system. The core principles of the invention remain applicable across different CAD platforms.

Object Grouping Mechanisms: A fundamental aspect of the invention is the utilization of object groupings within the CAD model to represent design intent and hierarchical relationships. Different CAD systems employ various mechanisms for grouping objects, all of which can be leveraged by the system.

\begin{itemize}
    \item Layers (Rhino): In Rhino, layers provide a primary means of organizing objects. The provided code explicitly uses layers (specifically sublayers denoted by "::") to group different parts of the chair model (e.g., "Chair::Legs," "Chair::Seat," "Chair::Backrest"). This layer-based organization directly corresponds to the concept of object groupings in this invention. The code demonstrates how materials are applied to objects based on their layer assignment, highlighting the importance of this hierarchical structure for generating variations.
    \item Components (Fusion 360): Fusion 360 utilizes components as a fundamental organizational structure. Similar to layers in Rhino, components can represent distinct parts or sub-assemblies within a design. The system can be adapted to utilize component information for object grouping and variation generation in Fusion 360.

    \item Groups/Bodies (SolidWorks): SolidWorks employs groups and bodies for organizing geometry. The system can be implemented to recognize these groupings and utilize them for applying variations and generating semantic labels.
\end{itemize}
The system is designed to be adaptable to different object grouping mechanisms. The core concept of organizing objects hierarchically to represent design intent remains consistent, regardless of the specific CAD software or its internal representation of groups.

\paragraph{Importing and Preprocessing CAD Data}
The process of importing CAD models into the system and any necessary preprocessing steps are crucial for ensuring compatibility and efficient data handling.

Direct Import: Ideally, the system directly imports the native file format of the chosen CAD software (e.g., .3dm for Rhino, .f3d for Fusion 360, .sldprt for SolidWorks). This direct import minimizes data loss and preserves the hierarchical structure of the model.

Mesh Conversion (Optional): In some cases, it may be necessary to convert the CAD geometry into a mesh representation for efficient rendering and processing. This conversion can be performed within the CAD software itself or using external libraries. The system handles mesh data appropriately, maintaining the object groupings and applying variations to the mesh elements. If mesh conversion is required, the system will ideally use the highest fidelity mesh representation available, balancing detail with computational efficiency.

Geometry Simplification (Optional): For very complex CAD models, geometry simplification techniques may be employed to reduce the computational load during variation generation and rendering. Simplification techniques, such as mesh decimation or NURBS curve reduction, can be applied while preserving the overall form and object groupings. The system incorporates parameters to control the level of simplification, allowing designers to balance detail with performance. The decision to simplify geometry is context-dependent and depends on the complexity of the model and the computational resources available. The system provides feedback to the designer if simplification is necessary and allows them to adjust the simplification parameters as needed.

\subsection{II. Automated Data Generation Module}

\paragraph{B. Automated Data Generation Module}

The Automated Data Generation Module is the core of the invention, responsible for creating the diverse and richly labeled dataset required to train the AI model.  It interacts directly with the CAD system, extracting object groupings, applying systematic variations, and generating corresponding semantic labels.

\subparagraph{Identifying Object Groupings:} This module employs algorithms to parse the CAD model and identify object groupings based on user-defined criteria or pre-defined rules.  This process leverages the hierarchical organization mechanisms within the CAD system.

    \begin{itemize}
        \item User-Defined Criteria: Designers can specify criteria for grouping objects, such as shared layer assignments, group memberships, or tag attributes. This allows for flexible and context-specific definition of object groupings that align with the designer's intent.

        \item Pre-defined Rules: For common design patterns, the system incorporates pre-defined rules for identifying object groupings. For instance, in furniture design, rules might be implemented to automatically group objects belonging to "legs," "seat," "backrest," etc., even if these objects are not explicitly grouped by the designer in the CAD model.

    \begin{verbatim}
      \#  Example code for identifying object groupings in Rhino
      The `get_objects_from_sublayers()` function iterates through all sublayers (identified by the "::" delimiter) and extracts the objects belonging to each sublayer. This layer-based grouping provides a direct mapping between the CAD model's organization and the system's understanding of design components.
    \end{verbatim}  

\subparagraph{Generating Variations:}  This module systematically applies variations to the CAD model across four key aspects: visual properties, camera viewpoints, lighting conditions, and backgrounds.

    \begin{itemize}
        \item Visual Properties (CMF): The system varies the color, material, and finish (CMF) properties of objects within the defined groupings.

        \begin{verbatim}
         \# Randomness and Predefined Ranges}
         The code example showcases the use of randomness and predefined ranges for material assignment. The `MATERIAL_LIST` defines a library of materials with their associated properties (name, RGB color, gloss, and reflectivity).  The `create_random_material()` function randomly selects a material from this list and applies it to the objects within a group. This randomized approach ensures a diverse range of CMF variations in the training dataset.  The system also allows for user-defined ranges for each CMF property, providing more control over the variation process.
        \end{verbatim}

        \item Procedural Materials: Beyond the predefined material library, the system can incorporate procedural material generation techniques. This allows for the creation of an infinite number of material variations, further enhancing the diversity of the training data.

        \item Material Blending:  The system can blend multiple materials together, creating complex and nuanced surface appearances.  This allows for more realistic and sophisticated representations of real-world materials.
    \end{itemize}

   \subparagraph{ \item }Camera Viewpoints:  The system varies the camera perspective to capture the design from different angles and distances.

\item
\begin{verbatim}
  \# Hero Shot and Orbital Representation
  Hero Shot and Orbital Representation:** The code provides examples of two camera viewpoint generation modes: "hero shot" and "orbital representation." The `generate_camera_positions()` function implements these modes. "Hero shot" generates camera positions clustered around a focal point, simulating product photography. "Orbital representation" generates camera positions along circular orbits around the object, providing a more comprehensive view of the design. The `variation_degrees` parameter controls the degree of randomness in camera positioning within each mode.
\end{verbatim}

\begin{itemize}
    \item \textbf{Focal Length Variation:} The \texttt{generate\_viewpoint\_description()} function demonstrates the variation of focal length, simulating different lens effects and perspectives. This adds another dimension to the viewpoint variations.
\end{itemize}

\textbf{c. Lighting Conditions:} The system generates diverse lighting scenarios, mimicking different environments and times of day.

\begin{itemize}
    \item \textbf{Lighting Parameters:} The \texttt{generate\_lighting\_variation()} function creates variations in lighting softness, direction, and intensity. This function provides a simplified example; the full system allows for more granular control over lighting parameters, including color temperature, shadows, and the type of light source (e.g., directional, point, spot, area).

    \item \textbf{Applying Lighting to Render:} The \texttt{apply\_lighting\_to\_render()} function (placeholder in the provided code) interacts with the Rhino rendering engine to apply the generated lighting settings. The full implementation utilizes the Rhino API to control various lighting parameters and create realistic lighting effects.
\end{itemize}

\textbf{d. Backgrounds:} The system applies different backgrounds to the rendered images, enhancing context and visual diversity.

\begin{itemize}
    \item \textbf{Background Library:} A library of predefined backgrounds, including solid colors, gradients, textures, patterns, and environmental scenes, is used to create variations.

    \item \textbf{Google Maps Integration:} The system integrates with Google Maps, allowing designers to use real-world locations as backgrounds. It utilizes depth maps derived from Google Maps data and ControlNet to ensure accurate placement and perspective of the 3D model within the chosen environment.
\end{itemize}

\item \textbf{Semantic Label Generation:} For each generated variation, the system creates a detailed semantic label.

\begin{itemize}
    \item \textbf{Structured Description:} The \texttt{generate\_caption()} function in the code provides a basic example of semantic label generation. The full system generates more comprehensive labels, incorporating information about:
    \begin{itemize}
        \item \textbf{Object Groupings and Materials:} "Chair legs made of brushed aluminum, seat upholstered in dark blue fabric, backrest made of light wood."
        \item \textbf{Camera Viewpoint:} "Viewed from a high angle, front right perspective, with a 50mm lens."
        \item \textbf{Lighting Conditions:} "Illuminated with soft, natural light from the left."
        \item \textbf{Background:} "Against a neutral gray background."
        \item \textbf{Design Intent (if provided by the user):} "Designed for a modern office environment."
    \end{itemize}
\end{itemize}

This structured approach ensures consistency and facilitates the AI model's understanding of the relationship between textual descriptions and visual attributes.

\item \textbf{Data Augmentation Techniques:} To maximize the diversity of the training dataset and improve the AI model's robustness, the system employs various data augmentation techniques:

\begin{itemize}
    \item \textbf{Random Cropping:} Randomly cropping sections of the rendered images.
    \item \textbf{Rotation and Flipping:} Rotating and flipping images to create variations in orientation.
    \item \textbf{Color Jitter:} Randomly adjusting brightness, contrast, saturation, and hue.
    \item \textbf{Noise Addition:} Adding small amounts of random noise to the images to simulate real-world imperfections.
\end{itemize}

These techniques artificially expand the training dataset, improving the AI model's ability to generalize and reducing the risk of overfitting. The specific data augmentation techniques used can be customized and adjusted based on the design context and the characteristics of the dataset.

In summary, the systematic variation of visual properties, camera viewpoints, lighting conditions, and backgrounds, combined with the automated generation of semantic labels, forms the foundation for training a robust and versatile AI model for design applications. The referenced code examples illustrate how these functionalities are implemented in practice, demonstrating the practicality and effectiveness of the invention.

\subsection{III. AI Model Training Module}
\section{Claims}

\subsection{TO DO:}
Expand claims to thoroughly cover the described features and functionalities, including the core innovation of form isolation and its application to preventing overfitting.)

\section{Conclusion}

This invention significantly advances the field of AI-assisted design by addressing key challenges in training data generation, workflow integration, and designer control. By focusing on form isolation and incorporating a comprehensive suite of advanced features, the system empowers designers to leverage the creative potential of AI while maintaining full control over their design intent and streamlining their workflow. The provided code examples offer a glimpse into the technical implementation of the system, demonstrating how the core functionalities are realized in practice. 

\end{document}