\documentclass{article}

\usepackage{graphicx}
\usepackage{hyperref}

\title{System and Method for Automated Training Data Generation for AI-Assisted Design Using Object Groupings and Semantic Labeling}

\begin{document}

\section{Abstract}

This invention presents a novel system and method for automated training data generation in AI-assisted design, leveraging object groupings and semantic labeling in CAD models. The system utilizes existing CAD functionalities to define hierarchical object relationships, facilitating semantic understanding and automated label generation. By systematically varying visual properties while maintaining core design forms, the system generates diverse training datasets that emphasize form recognition. Automated semantic label generation captures object groupings, applied variations, and design intent. The system employs Low-Rank Adaptation (LoRA) for efficient AI model training and includes an advanced user interface for enhanced design exploration and workflow optimization.

\section{Background of the Invention}

The field of AI-assisted design has seen significant advancements in recent years, yet challenges persist in creating efficient workflows that seamlessly integrate AI capabilities with traditional Computer-Aided Design (CAD) processes. Existing systems often struggle to generate diverse and semantically rich training data, limiting the AI's ability to understand and manipulate complex design elements effectively.

Current CAD workflows typically involve manual processes for organizing design elements, applying variations, and generating training data for AI models. This manual approach is time-consuming, prone to inconsistencies, and often fails to capture the full range of design possibilities. Moreover, existing AI-assisted design tools frequently lack the ability to understand hierarchical object relationships and design intent, leading to suboptimal results and increased designer workload.

\section{Description of the Related Art}

Several attempts have been made to address the challenges in AI-assisted design:

1. Patent US10885440B2 describes a system for generating synthetic data for machine learning models in computer vision applications. While this system offers methods for data augmentation, it does not specifically address the unique requirements of CAD-based design workflows or the need for semantic understanding of design elements.

2. Patent US20200334456A1 presents a method for automated 3D model generation using machine learning. However, this approach focuses primarily on generating 3D models from 2D inputs and does not address the need for diverse training data generation from existing CAD models.

3. Research paper "Deep learning for CAD data retrieval and 3D shape classification" by Furuya et al. (2020) explores the use of deep learning for CAD model retrieval and classification. While this work demonstrates the potential of AI in CAD applications, it does not provide a comprehensive solution for training data generation or semantic labeling in design workflows.

The present invention addresses the limitations of these existing approaches by providing a unified system that leverages CAD object groupings, systematic variation, and semantic labeling to generate rich, diverse training datasets while streamlining the AI-assisted design process.

\section{Summary of the Invention}

The present invention provides a system and method for automated training data generation in AI-assisted design, utilizing object groupings and semantic labeling in CAD models. Key components and functionalities include:

\subsection{Object Grouping in CAD Models}
The system leverages existing CAD functionalities such as layers, groups, and tags to define hierarchical object relationships that reflect design intent. This structured organization facilitates semantic understanding and automated label generation, mirroring the way designers conceptualize their creations.

\subsection{Form Isolation through Systematic Variation}
The invention generates diverse variations of CAD models by systematically altering visual properties, camera viewpoints, lighting conditions, and backgrounds while maintaining the core design form as the invariant element. This approach enables the AI model to learn form independently of other visual attributes, preventing overfitting and generating a robust training dataset.

\subsection{Automated Semantic Label Generation}
The system automatically creates descriptive text labels for each generated variation, capturing object groupings, applied variations, and design intent. These labels provide the necessary textual context for training the AI model and facilitating natural language interaction with the system.

\subsection{AI Model Training using LoRA}
The invention employs Low-Rank Adaptation (LoRA) to fine-tune pre-trained text-to-image AI models efficiently. This method adapts the model to specific design data without retraining the entire model, preserving computational resources and accelerating the training process.

\subsection{AI-Assisted Design Interface with Advanced Features}
A user-friendly interface enables designers to interact with the trained AI model through natural language prompts. The interface includes advanced features such as automated file management, detailed close-ups, location-based backgrounds, mass and volume calculations, design callouts, character continuity, dynamic UI sliders, patent drawing generation, design analysis, client annotation tools, image security, vision tracking, modular UI, style references, thematic sliders, preset management, export options, and history tree with visual diff functionality.

\section{Brief Description of the Drawings}

Figure 1: System architecture diagram illustrating the key components and data flow of the automated training data generation system.

Figure 2: Flowchart depicting the process of object grouping and hierarchical relationship definition in CAD models.

Figure 3: Diagram showing the systematic variation process, including examples of form isolation through changes in materials, lighting, and viewpoints.

Figure 4: Illustration of the automated semantic label generation process, with examples of generated labels for various design variations.

Figure 5: Schematic representation of the LoRA-based AI model training process.

Figure 6: User interface mockup showcasing the AI-assisted design interface and its advanced features.

Figures 7-10: Detailed illustrations of specific UI elements and functionalities, including the modular UI, thematic sliders, and history tree with visual diff.

\section{Detailed Description of the Invention}

\subsection{Object Grouping in CAD Models}

The present invention leverages existing CAD functionalities to define hierarchical object relationships that reflect design intent. This process involves:

1. Utilizing CAD layers, groups, and tags to organize model components.
2. Establishing parent-child relationships between objects to represent design hierarchy.
3. Applying semantic tags to object groups to capture design intent and functionality.

For example, in a chair design, the system might organize components as follows:

- Chair (root object)
  - Seat
    - Cushion
    - Frame
  - Backrest
    - Support
    - Padding
  - Legs
    - Front Left
    - Front Right
    - Back Left
    - Back Right

This hierarchical structure enables the system to understand the relationships between components and apply variations or generate labels accordingly.

\subsection{Form Isolation through Systematic Variation}

The system generates diverse variations of the CAD model while maintaining the core design form as the invariant element. This process includes:

1. Identifying the core form elements that should remain constant.
2. Systematically altering visual properties such as:
   - Colors, materials, and finishes (CMF)
   - Camera viewpoints and perspectives
   - Lighting conditions (intensity, color, direction)
   - Backgrounds and environments

The variation process is controlled by parameters that ensure a wide range of visually distinct outputs while preserving the essential design characteristics.

\subsection{Automated Semantic Label Generation}

For each generated variation, the system automatically creates descriptive text labels. The label generation process involves:

1. Extracting information from the CAD model's hierarchical structure.
2. Incorporating details about applied variations (materials, lighting, viewpoint).
3. Capturing design intent based on semantic tags and object relationships.

A typical generated label might look like:

"Modern office chair with ergonomic backrest, viewed from a 45-degree angle. Matte black metal frame with light gray fabric upholstery. Soft ambient lighting from above."

These labels provide the necessary context for training the AI model and enable natural language interaction with the system.

\subsection{AI Model Training using LoRA}

The system employs Low-Rank Adaptation (LoRA) to fine-tune pre-trained text-to-image AI models efficiently. This method adapts the model to specific design data without retraining the entire model, preserving computational resources and accelerating the training process.

\subsection{AI-Assisted Design Interface with Advanced Features}

A user-friendly interface enables designers to interact with the trained AI model through natural language prompts. The interface includes advanced features such as:

\subsubsection{Automated File Management}
The system automates file organization and storage in the cloud, simplifying asset management and facilitating collaboration.

\subsubsection{Detailed Close-Ups}
The system generates high-resolution close-up renders of specific design elements.

\subsubsection{Location-Based Backgrounds}
The system incorporates real-world locations as backgrounds, enhancing realism and context. This feature integrates with Google Maps API, depth map generation, and ControlNet for precise background placement and control.

\subsubsection{Mass \& Volume Calculations}
The system calculates mass, volume, estimated cost, and environmental impact based on CMF properties and CAD data. It also generates automated RFQs for materials and components.

\subsubsection{Design Callouts}
The system automatically generates professional-style design callouts using image segmentation and prompt generation. Users can edit and customize the callouts as needed.

\subsubsection{Character Continuity}
The system maintains character consistency across multiple renders by implementing techniques such as face swapping and text-based character description analysis.

\subsubsection{Dynamic UI Sliders}
The system dynamically generates sliders based on the prompt context, providing intuitive control over rendering parameters.

\subsubsection{Patent Drawing Generation}
The system automates the creation of patent drawings from the CAD model, including line art generation and annotation placement.

\subsubsection{Design Analysis}
The system integrates with a GPT-based model for design analysis. It identifies potential problems and areas for improvement, providing valuable feedback to designers.

\subsubsection{Client Annotation Tools}
The system facilitates collaborative annotation by allowing clients and reviewers to provide feedback directly on rendered images.

\subsubsection{Image Security}
The system protects intellectual property and tracks image usage through watermarking and serialization techniques.

\subsubsection{Vision Tracking}
The system implements eye-tracking analysis and generates heatmaps to visualize user attention patterns.

\subsubsection{Modular UI}
The system offers a flexible and customizable interface for applying materials and styles. It includes drag-and-drop functionality and texture blending capabilities.

\subsubsection{Style References}
The system allows designers to apply different artistic styles to their renders. It provides control over style parameters and supports style transfer techniques.

\subsubsection{Thematic Sliders}
The system includes sliders for controlling mood and overall style, enabling designers to adjust the aesthetic of the render.

\subsubsection{Preset Management}
The system enables efficient reuse of rendering settings and styles through a preset management system. Users can save, load, and share rendering configurations.

\subsubsection{Export Options}
The system supports various output formats for different applications, such as 3D printing, foldable models, and more.

\subsubsection{History Tree and Visual Diff}
The system tracks design iterations and provides visual comparison capabilities, allowing designers to compare and analyze variations.

\section{Claims}

TO DO: Draft clear and concise claims that cover both the broad aspects and specific features of the invention.

\section{Conclusion}

The present invention provides a comprehensive solution for automated training data generation in AI-assisted design, addressing the limitations of existing approaches. By leveraging object groupings, systematic variation, and semantic labeling in CAD models, the system generates rich, diverse training datasets that enable the AI model to understand and manipulate complex design elements effectively. The advanced user interface and workflow optimization features streamline the design process, enhancing creativity and efficiency.

\end{document}

\textbf{c. Lighting Conditions:} The system generates diverse lighting scenarios, mimicking different environments and times of day.

\begin{itemize}
    \item \textbf{Lighting Parameters:} The \texttt{generate\_lighting\_variation()} function creates variations in lighting softness, direction, and intensity. This function provides a simplified example; the full system allows for more granular control over lighting parameters, including color temperature, shadows, and the type of light source (e.g., directional, point, spot, area).

    \item \textbf{Applying Lighting to Render:} The \texttt{apply\_lighting\_to\_render()} function (placeholder in the provided code) interacts with the Rhino rendering engine to apply the generated lighting settings. The full implementation utilizes the Rhino API to control various lighting parameters and create realistic lighting effects.
\end{itemize}

\textbf{d. Backgrounds:} The system applies different backgrounds to the rendered images, enhancing context and visual diversity.

\begin{itemize}
    \item \textbf{Background Library:} A library of predefined backgrounds, including solid colors, gradients, textures, patterns, and environmental scenes, is used to create variations.

    \item \textbf{Google Maps Integration:} The system integrates with Google Maps, allowing designers to use real-world locations as backgrounds. It utilizes depth maps derived from Google Maps data and ControlNet to ensure accurate placement and perspective of the 3D model within the chosen environment.
\end{itemize}

\item \textbf{Semantic Label Generation:} For each generated variation, the system creates a detailed semantic label.

\begin{itemize}
    \item \textbf{Structured Description:} The \texttt{generate\_caption()} function in the code provides a basic example of semantic label generation. The full system generates more comprehensive labels, incorporating information about:
    \begin{itemize}
        \item \textbf{Object Groupings and Materials:} "Chair legs made of brushed aluminum, seat upholstered in dark blue fabric, backrest made of light wood."
        \item \textbf{Camera Viewpoint:} "Viewed from a high angle, front right perspective, with a 50mm lens."
        \item \textbf{Lighting Conditions:} "Illuminated with soft, natural light from the left."
        \item \textbf{Background:} "Against a neutral gray background."
        \item \textbf{Design Intent (if provided by the user):} "Designed for a modern office environment."
    \end{itemize}
\end{itemize}

This structured approach ensures consistency and facilitates the AI model's understanding of the relationship between textual descriptions and visual attributes.

\item \textbf{Data Augmentation Techniques:} To maximize the diversity of the training dataset and improve the AI model's robustness, the system employs various data augmentation techniques:

\begin{itemize}
    \item \textbf{Random Cropping:} Randomly cropping sections of the rendered images.
    \item \textbf{Rotation and Flipping:} Rotating and flipping images to create variations in orientation.
    \item \textbf{Color Jitter:} Randomly adjusting brightness, contrast, saturation, and hue.
    \item \textbf{Noise Addition:} Adding small amounts of random noise to the images to simulate real-world imperfections.
\end{itemize}

These techniques artificially expand the training dataset, improving the AI model's ability to generalize and reducing the risk of overfitting. The specific data augmentation techniques used can be customized and adjusted based on the design context and the characteristics of the dataset.

In summary, the systematic variation of visual properties, camera viewpoints, lighting conditions, and backgrounds, combined with the automated generation of semantic labels, forms the foundation for training a robust and versatile AI model for design applications. The referenced code examples illustrate how these functionalities are implemented in practice, demonstrating the practicality and effectiveness of the invention.

\subsection{III. AI Model Training Module}


%This module is responsible for training the AI model that will ultimately empower designers to generate renderings from natural language prompts. It leverages a pre-trained text-to-image model as a foundation and fine-tunes it using the dataset generated by the Automated Data Generation Module. This approach combines the general image generation capabilities of a large pre-trained model with the specific design knowledge encapsulated in the custom dataset.



This module is the heart of the invention, responsible for imbuing the AI with the ability to understand and generate variations of the designer's intended form. Instead of merely learning to associate specific visual styles with a design, the AI is trained to recognize and reproduce the underlying form of the design as the primary element, independent of potentially distracting variations in color, material, texture, lighting, or viewpoint. This is achieved by leveraging the meticulously curated dataset generated by the Automated Data Generation Module, which embodies the core principle of form isolation.

\subsubsection{Pre-trained Model Selection}
The system starts with a pre-trained text-to-image AI model, chosen from a range of suitable open-source options. The key criteria for model selection are:

\begin{itemize}
    \item \textbf{Open-Source License:} Essential for maintaining complete ownership and control over the technology, allowing for unrestricted customization, private deployment, and future development.
    \item \textbf{Architectural Flexibility:} The model's architecture should be adaptable for fine-tuning, ideally through efficient techniques like Low-Rank Adaptation (LoRA), which allow for targeted modification without retraining the entire model.
    \item \textbf{Adequate Image Quality:} The model should be capable of generating images with sufficient quality and resolution for design visualization purposes. However, perfect photorealism is not necessarily the primary goal at this stage; the focus is on the model's ability to learn and represent form.
\end{itemize}

Examples of potential open-source models include Stable Diffusion, or other emerging diffusion-based models known for their flexibility and efficiency. The specific choice of model is not critical to the invention; rather, it's the subsequent training process that imbues the model with the unique capability to understand and generate design forms independently of stylistic variations.

\subsubsection{Training for Form Recognition, Not Style Matching}
The heart of this invention lies in how the AI model is trained. Unlike conventional approaches that focus on teaching the AI to mimic specific visual styles or aesthetics, this system trains the AI to recognize and reproduce the underlying form of the design as the primary element. This is achieved by leveraging the unique dataset generated by the Automated Data Generation Module, which explicitly embodies the form isolation principle:

\paragraph{Dataset Structure:} The dataset consists of numerous variations of the same design, where the form remains constant while all other visual attributes (CMF properties, camera viewpoints, lighting, backgrounds) are systematically and randomly varied. This forces the AI to focus on the only consistent element across all images: the form itself.

\paragraph{Semantic Label Guidance:} The semantic labels accompanying each image reinforce this focus on form. Instead of merely describing the specific visual features present in each image, the labels emphasize the underlying form and its constituent parts. For example, a label might read: "A chair with [varied material] legs, a [varied material] seat, and a [varied material] backrest, viewed from a [varied angle]." This structured labeling, combined with the dataset's visual emphasis on form, guides the AI to learn a representation of the form that is decoupled from specific stylistic choices.

\subsubsection{Fine-tuning with LoRA}
Low-Rank Adaptation (LoRA) is a highly efficient fine-tuning technique that allows the pre-trained model to learn from the custom dataset without retraining the entire model from scratch. This is particularly beneficial when working with large, complex models:

\begin{itemize}
    \item \textbf{Targeted Modification:} LoRA injects small, low-rank matrices into specific layers of the pre-trained model, allowing for targeted modifications that specialize the model's behavior without disrupting its general capabilities. This ensures that the model retains its ability to understand and generate images from text prompts, while also incorporating the unique knowledge gained from the form-focused dataset.
    \item \textbf{Preventing Overfitting:} LoRA's efficiency is particularly valuable in this context. Because the training dataset consists of numerous variations of the same form, there is a risk of the model overfitting to the specific variations present in the dataset. LoRA's ability to adapt the model with minimal parameter changes mitigates this risk, ensuring that the model learns the general concept of the form rather than memorizing the specific variations seen during training.
\end{itemize}

\subsubsection{Evaluation and Validation}
The training process is closely monitored and evaluated to ensure that the model is effectively learning to recognize and generate the intended form.

\begin{itemize}
    \item \textbf{Visual Inspection:} Designers play a crucial role in evaluating the model's outputs. They visually inspect the generated images to assess their fidelity to the original form, ensuring that the AI is capturing the essential design elements and relationships.
    \item \textbf{Quantitative Metrics:} Objective metrics like Fréchet Inception Distance (FID) and CLIP Score are used to measure the quality and relevance of the generated images. However, these metrics are interpreted in the context of the invention's specific goal: The focus is not solely on generating photorealistic images, but rather on ensuring that the generated images accurately represent the intended form, regardless of the stylistic variations applied.
    \item \textbf{Generalization Tests:} The model is tested on unseen prompts and variations to assess its ability to generalize beyond the specific examples seen during training. This ensures that the AI can effectively generate novel variations of the form, responding to designer prompts that specify new CMF properties, lighting conditions, or viewpoints, while still preserving the essential design characteristics.
\end{itemize}

This revised description of the AI Model Training Module clarifies the crucial distinction between training for form recognition and conventional style matching. It emphasizes how the form-isolated dataset, combined with targeted fine-tuning using LoRA, enables the AI to learn and generalize the designer's intended form as the primary element, decoupling it from specific stylistic variations. This approach is central to the invention's ability to empower designers with unprecedented control over visual attributes while maintaining the integrity of their original design.

\subsection{IV. AI-Assisted Design Interface}

This component serves as the bridge between the designer's intent and the AI's rendering capabilities. It's a user-friendly interface that empowers designers to interact with the trained AI model through natural language, leveraging its unique ability to generate variations of a design while preserving its core form. The interface goes beyond basic text-to-image functionality, incorporating a suite of advanced features designed to streamline the design workflow, enhance creative exploration, and facilitate collaboration.

\subsubsection{Core Rendering Functionality}

\textbf{Natural Language Prompts}: Designers interact with the AI using natural language descriptions. They can specify desired changes to the CMF properties of the design, adjust lighting conditions, request specific camera viewpoints, and even incorporate elements like characters or backgrounds.

\textbf{Prompt Parsing and Parameter Translation}: The interface parses the designer's text prompt, identifying key elements and translating them into rendering parameters that the AI model can understand. This involves natural language processing techniques to extract relevant information, such as:

\begin{itemize}
\item \textbf{Object References}: Identifying which parts of the design the prompt refers to (e.g., "the legs," "the seat," "the backrest"). This leverages the object groupings defined in the CAD model.
\item \textbf{CMF Properties}: Extracting desired changes to color, material, and finish. For example, "make the legs brushed aluminum," "upholster the seat in blue fabric," or "give the backrest a wood grain texture."
\item \textbf{Lighting and Viewpoint}: Interpreting descriptions of lighting conditions (e.g., "soft natural light," "dramatic spotlight") and camera viewpoints (e.g., "front view," "isometric perspective").
\end{itemize}

\textbf{Rendering Generation}: The AI model, guided by the extracted parameters, generates a rendering of the design, incorporating the specified changes while preserving the integrity of the original form. The model's ability to separate form from style, as trained by the form-isolated dataset, is crucial here. It allows designers to freely manipulate visual attributes without altering the fundamental design.

\subsubsection{Advanced Features}

Beyond core rendering functionality, the interface offers a rich set of advanced features to enhance the design process:

\subsubsection{Automated File Management}
\begin{itemize}
\item \textbf{Automated Naming}: The system generates descriptive file names for renderings based on the prompt content, eliminating manual naming and ensuring easy identification. For example, a prompt like "chair with red fabric seat, side view, warm lighting" might result in a file name like \textit{Chair_RedFabricSeat_SideView_WarmLighting.jpg}.
\item \textbf{Cloud Storage}: Renderings and associated data are automatically stored in the cloud, providing secure access from anywhere and facilitating collaboration. Version control is integrated, allowing designers to track changes and revert to previous versions if needed.
\item \textbf{Archiving}: Projects and related data can be archived for future reference, ensuring easy retrieval and organization.
\end{itemize}

\subsubsection{Detailed Close-Ups}
\begin{itemize}
\item \textbf{Region Selection}: Designers can specify regions of interest within the prompt or directly on the rendered image to generate high-resolution close-ups of specific design elements. For instance, a prompt could include "generate a detailed close-up of the joinery details on the armrest."
\item \textbf{Resolution Control}: The system allows for adjustable resolution settings for close-up renders, enabling designers to create highly detailed visualizations for presentations or manufacturing purposes. This could be implemented as a simple slider in the UI, allowing the user to select a desired resolution for the close-up.
\end{itemize}

\subsubsection{Location-Based Backgrounds}
\begin{itemize}
\item \textbf{Google Maps Integration}: The system seamlessly integrates with Google Maps, allowing designers to select real-world locations as backgrounds for their renderings. The user can specify a location and viewpoint directly within the prompt (e.g., "place the bench in Central Park, facing the Bethesda Terrace," as depicted in Figure \ref{fig:location-based-background}).
\item \textbf{Depth Map Generation}: The system automatically retrieves depth information from Google Maps data for the chosen location. This depth data is used to create a 3D representation of the environment, ensuring accurate placement and perspective of the design within the background.
\item \textbf{ControlNet}: ControlNet is employed to precisely position the design object within the 3D environment created from the Google Maps data. This ensures that the perspective, scale, and occlusion relationships between the design and the background are rendered realistically.
\end{itemize}

\begin{figure}
\centering
\includegraphics[width=0.8\linewidth]{images/C1.jpg}
\caption{Illustrative Embodiment: Location-Based Backgrounds. The user can specify a location and viewpoint in the prompt, and the system uses Google Maps data and ControlNet to generate a realistic rendering with the design placed in the chosen environment.}
\label{fig:location-based-background}
\end{figure}

\subsubsection{Mass \& Volume Calculations}
\begin{itemize}
\item \textbf{Automated Analysis}: Based on the dimensional information from the CAD model and the CMF properties specified in the prompt, the system automatically calculates the estimated weight, volume, and material costs of the design. For example, if the prompt specifies "brushed stainless steel" for a bench frame, the system will calculate the volume of the frame, reference the density of stainless steel, and then calculate the estimated weight and cost based on current market prices.
\item \textbf{Environmental Impact}: The system estimates the environmental impact of the design based on the chosen materials and manufacturing processes. This could involve calculating the carbon footprint based on material extraction, processing, and transportation data. 
\item \textbf{Automated RFQ Generation}: The system can automatically generate requests for quotations (RFQs) based on the calculated material requirements and manufacturing specifications. This could involve automatically filling out web forms or generating emails pre-populated with the relevant design and material information, streamlining the process of obtaining quotes from suppliers.
\end{itemize}

\subsubsection{Design Callouts}
\begin{itemize}
\item \textbf{Automated Callout Generation}: The system can automatically generate professional-style design callouts on the rendered images. These callouts use image segmentation techniques to identify relevant design elements and associate them with text annotations based on the prompt or predefined rules. For example, a prompt like "highlight the ergonomic features" could trigger callouts pointing to the curved backrest, adjustable lumbar support, and padded armrests, with corresponding annotations describing these features. In conjunction with the enviormental analysis module previously discussed, callouts with kg/CO2e can then be generated for the drawing to help the design be aware of the potential enviormental consequences of their material choices.
\item \textbf{Editable Annotations}: Designers have full control over the generated callouts. They can edit the text, adjust the positioning, change the style, and add or remove callouts as needed. This allows for precise and personalized annotation of the design.
\item \textbf{Searchable PDF Export}: When exporting the rendering as a PDF, the callouts are embedded as searchable text elements, facilitating easy navigation and reference. This ensures that the annotations are not just visual elements but also part of the document's searchable content.
\end{itemize}

\subsubsection{Character Continuity}
\begin{itemize}
\item \textbf{Consistent Character Appearance}: When rendering scenes involving human figures or characters, the system ensures consistent character appearance across multiple variations. This is particularly important when showing a character interacting with the design in different scenarios or using variations to refine the scene's composition.
\item \textbf{Face Swapping}: The system employs face-swapping techniques to maintain a consistent facial appearance for the chosen character, even as other aspects of the scene (e.g., clothing, pose, lighting) vary. This allows the designer to experiment with different variations without needing to re-generate or manually adjust the character's face in each rendering.
\item \textbf{Text-Based Description Matching}: The system analyzes the text prompts for descriptions of the character (e.g., "a woman in her mid-40s with short blonde hair, wearing a business suit"). It uses these descriptions to ensure that the character's appearance consistently matches the designer's intent across different renders, even if the prompt only mentions the character briefly (e.g., "show the woman sitting on the bench").
\end{itemize}

\subsubsection{Dynamic UI Sliders}
\begin{itemize}
\item \textbf{Context-Sensitive Sliders}: The interface dynamically generates sliders for controlling rendering parameters based on the context of the text prompt. For example, if the prompt mentions "a bench against a backdrop of a stormy sky and waves," sliders would appear for controlling the intensity of the storm, the turbulence of the waves, the brightness of the lightning, as well as more conventional parameters like time of day (see Figure \ref{fig:dynamic-sliders}).
\item \textbf{Intuitive Control}: These sliders provide a visual and intuitive way for designers to fine-tune the rendering without needing to modify the text prompt directly. They allow for a more granular and interactive exploration of the design space. This interactivity can speed up the design process, allowing for rapid experimentation and refinement.
\item \textbf{Parameter Mapping}: The system maps the slider values to specific rendering parameters in the AI model, ensuring that the slider adjustments are accurately reflected in the generated image. This mapping could be linear, non-linear, or even based on learned relationships between slider values and desired visual effects, depending on the complexity of the parameter and its impact on the rendering.
\end{itemize}

\begin{figure}
\centering
\includegraphics[width=0.8\linewidth]{images/C1.jpg}
\caption{Illustrative Embodiment: Dynamic UI Sliders. Context-sensitive sliders appear based on the prompt, allowing for granular control over elements like storm intensity, wave turbulence, and lightning brightness.}
\label{fig:dynamic-sliders}
\end{figure}

\subsubsection{Patent Drawing Generation}
\begin{itemize}
\item \textbf{Automated Line Art}: The system can automatically generate line art drawings from the CAD model suitable for patent applications. It utilizes edge detection algorithms and vectorization techniques to create clean, precise line drawings that meet the requirements of patent offices (see Figure \ref{fig:patent-drawing}).
\item \textbf{Annotation Placement}: Annotations such as dimension lines, reference numerals, and technical labels are automatically placed according to patent drawing conventions. The system uses the structured data from the CAD model to accurately determine dimensions and relationships between design elements, ensuring that the annotations are placed correctly and comply with patent regulations.
\item \textbf{Export Format Compliance}: The generated patent drawings are exported in a format compliant with the specific requirements of the target patent office, simplifying the patent filing process.
\end{itemize}

\begin{figure}
\centering
\includegraphics[width=0.8\linewidth]{images/B2.jpg}
\caption{Illustrative Embodiment: Patent Drawing Generation. The system automatically generates line art drawings with annotations, conforming to patent drawing standards.}
\label{fig:patent-drawing}
\end{figure}

\subsubsection{Design Analysis}
\begin{itemize}
\item \textbf{GPT Integration}: The system integrates with a large language model (LLM) like GPT-3 or a specialized design-focused LLM. This LLM is used to analyze the design in the context of the prompt and identify potential issues or areas for improvement. For example, if the prompt specifies "a park bench made of unfinished hickory wood in West Virginia," the LLM could access external knowledge bases to determine that hickory is susceptible to carpenter ant infestations in that region, flagging a potential durability issue.
\item \textbf{Problem Highlighting}: The LLM analyzes the design for factors like material suitability, structural integrity, ergonomics, and manufacturing feasibility. It flags potential problems and provides feedback to the designer, highlighting areas that might require further consideration or refinement. This feedback can be presented as text-based warnings, visual highlights on the rendered image, or even suggested modifications to the design or materials.
\item \textbf{Contextual Feedback}: The design analysis takes into account the specific context described in the prompt. For example, if the prompt mentions "outdoor furniture," the LLM might check for weather resistance and UV degradation of the chosen materials. The more context the designer provides in the prompt, the more refined and relevant the design analysis feedback will be.
\end{itemize}

\subsubsection{Client Annotation Tools}
\begin{itemize}
\item \textbf{Collaborative Review}: The interface provides tools for clients and reviewers to add annotations, comments, and feedback directly on the rendered images, as illustrated in Figure \ref{fig:client-annotation}. This facilitates a more streamlined and collaborative review process, eliminating the need for separate feedback documents or email chains.
\item \textbf{Visual Feedback}: Clients can draw directly on the image, highlight specific areas, add text boxes, or use pre-defined stamps (e.g., "approve," "revise"). This visual feedback is more intuitive and efficient than text-based comments alone, allowing clients to clearly communicate their thoughts and preferences.
\item \textbf{Revision Tracking}: The system tracks all annotations and comments, creating a clear record of the feedback and facilitating iterative revisions. Designers can easily see which areas have received feedback and address specific concerns raised by the client.
\end{itemize}

\begin{figure}
\centering
\includegraphics[width=0.8\linewidth]{images/C1.jpg}
\caption{Illustrative Embodiment: Client Annotation Tools. Clients can add comments, draw on the image, or use stamps to provide direct feedback on the rendering.}
\label{fig:client-annotation}
\end{figure}

\subsubsection{Image Security}
\begin{itemize}
\item \textbf{Watermarking}: Invisible watermarks are embedded within generated images to protect the designer's intellectual property. These watermarks can contain information about the designer, the project, or the date of creation, and can be used to track the origin of the image if it is distributed without authorization.
\item \textbf{Serialization}: Each rendered image is assigned a unique serial number that can be used to track its usage and distribution. This serial number can be linked to specific client information, allowing the designer to monitor who has access to which images.
\item \textbf{Usage Monitoring}: The system can detect if a watermarked image is used without authorization or if it appears in unauthorized locations online. This monitoring could be automated using image recognition techniques and web crawling tools, providing alerts to the designer if unauthorized use is detected.
\end{itemize}

\subsubsection{Vision Tracking \& Form Review}
\begin{itemize}
\item \textbf{Eye-Tracking Integration}: The system can optionally integrate with eye-tracking technology, using the embedded webcam on a laptop or the rear-facing camera on a mobile phone, for example, to analyze user gaze patterns during the review process.
\item \textbf{Heatmap Visualization}: Eye-tracking data is visualized as heatmaps overlaid on the rendered images for the designer to review. The heatmap can indicate which parts of the design are calling the viewer's attention. The degree to which the viewer's attention aligns with the intended hierarchy of design elements can be quantified, providing valuable insights to the designer. It allows for an objective assessment of whether the viewer is noticing the intended focal points of the design or if their attention is drawn to unintended areas. This feedback can guide the designer in refining the composition, proportions, repetition, and contrast of the design elements to achieve the desired visual impact.
\end{itemize}

\subsubsection{Modular UI}
\begin{itemize}
\item \textbf{Drag-and-Drop Interface}: Designers can drag and drop images or textures onto the design to apply materials or patterns, creating a more intuitive and direct way of customizing the rendering. For example, a designer could drag an image of a specific wood grain onto the bench seat to apply that texture, or drag a color swatch from a palette onto the legs to change their color.
\item \textbf{Texture Blending}: The system supports blending multiple textures together to create more complex and nuanced material effects. For instance, a designer could blend a rough stone texture with a polished metal overlay to create a unique surface finish, allowing for a greater degree of creative control over the material appearance.
\end{itemize}

\subsubsection{Style References}
\begin{itemize}
\item \textbf{Style Transfer}: Designers can apply different artistic styles to renderings by providing reference images. For example, dragging and dropping a still frame from a Pixar movie could instruct the system to render the design in a similar style, even if the prompt describes realistic materials. This allows designers to explore a wide range of aesthetics beyond photorealism.
\item \textbf{Dynamic Style Parameters}: Sliders control the strength and intensity of the applied style. This allows for fine-grained control over the style transfer effect, enabling the designer to blend the reference style with the original rendering to achieve the desired look.
\end{itemize}

\subsubsection{Thematic Sliders}
\begin{itemize}
\item \textbf{Mood and Atmosphere}: Sliders control the overall mood and atmosphere of the rendering (see Figure \ref{fig:thematic-sliders}). For example, sliders for "brightness," "contrast," "saturation," "warmth," "sharpness," and "depth of field" allow the designer to create a specific emotional tone or visual style without needing to describe these qualities in the text prompt.
\item \textbf{Style Presets}: Thematic sliders can be combined into presets, allowing designers to quickly apply predefined stylistic looks to their renderings (e.g., "cinematic," "retro," "minimalist").
\end{itemize}

\begin{figure}
\centering
\includegraphics[width=0.8\linewidth]{images/B3.jpg}
\caption{Illustrative Embodiment: Thematic Sliders. Sliders provide control over mood and atmosphere, allowing designers to easily create different stylistic feels.}
\label{fig:thematic-sliders}
\end{figure}

\subsubsection{Preset Management}
\begin{itemize}
\item \textbf{Saving and Loading Presets}: Designers can save their preferred rendering configurations as presets, including CMF choices, lighting settings, camera viewpoints, style references, and thematic slider settings. This allows for the reuse of successful rendering styles across different projects or design variations.
\item \textbf{Sharing Presets}: Presets can be shared among team members or publicly, fostering consistency and collaboration in design visualization. This can be particularly useful for maintaining a consistent brand identity or visual style across a design team.
\end{itemize}

\subsubsection{Export Options}
\begin{itemize}
\item \textbf{Image Formats}: The system supports exporting renders in various image formats, including JPG, PNG, TIFF, and EXR, catering to different resolution and quality requirements. This flexibility ensures compatibility with a wide range of design workflows and presentation formats.
\item \textbf{3D Printable Models}: Designers can export the design as a 3D printable model in formats like STL or OBJ, allowing for rapid prototyping and physical evaluation of the design.
\item \textbf{Foldable Paper Models}: The system can generate patterns for creating foldable paper models of the design, providing a tangible and interactive way to explore the form. This can be especially useful in the early stages of design development or for educational purposes.
\item \textbf{Vector Graphics}: Designers can export the rendering as vector graphics (e.g., SVG), enabling scalability and integration with other design software. Vector graphics are particularly useful for creating illustrations, logos, and other design assets that need to be resized without losing quality.
\end{itemize}

\subsubsection{History Tree and Visual Diff}
\begin{itemize}
\item \textbf{Iteration Tracking}: The interface maintains a history of all design iterations, including changes to the prompt, applied materials, lighting settings, and camera viewpoints. This provides a clear record of the design process and allows designers to easily revert to previous versions if needed. This history can be visualized as a tree structure or timeline, providing a clear overview of the design's evolution.
\item \textbf{Visual Diff}: The system highlights the differences between two selected versions of the rendering, visually showing the changes that have been made. This can be implemented by overlaying the two versions and highlighting the areas that have changed, making it easy for the designer to identify the impact of specific modifications. This feature facilitates a clear understanding of the design evolution and aids in iterative refinement.
\end{itemize}

This AI-Assisted Design Interface, with its core rendering functionality and extensive suite of advanced features, empowers designers to leverage the power of AI for creative exploration, efficient workflow, and effective communication. By focusing on natural language interaction, form preservation, and a user-centered design, the interface creates a seamless and intuitive experience that enhances the designer's creative process and unlocks new possibilities in design visualization. The integrated file management, analysis tools, collaborative features, and security measures further streamline the design workflow and address the practical needs of professional designers.

\section{Claims}

\subsection{TO DO:}
Expand claims to thoroughly cover the described features and functionalities, including the core innovation of form isolation and its application to preventing overfitting.)

\section{Conclusion}

This invention significantly advances the field of AI-assisted design by addressing key challenges in training data generation, workflow integration, and designer control. By focusing on form isolation and incorporating a comprehensive suite of advanced features, the system empowers designers to leverage the creative potential of AI while maintaining full control over their design intent and streamlining their workflow. The provided code examples offer a glimpse into the technical implementation of the system, demonstrating how the core functionalities are realized in practice. 

\end{document}